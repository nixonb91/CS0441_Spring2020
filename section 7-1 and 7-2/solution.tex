\documentclass[12pt]{article}
\usepackage[margin=1in]{geometry}
\usepackage{relsize}
\usepackage{amssymb}
\usepackage{amsmath}
\usepackage{enumitem}

\hyphenpenalty=10000
\setlength{\parindent}{0pt}

\newcommand\bicond{\mathrel{\leftrightarrow}}
\let\union\cup
\let\intersect\cap

% define environment for answers
\usepackage{xcolor}
\newenvironment{answer}{\larger[2]}{}

\begin{document}

\begin{center}
CS 441: Discrete Structures for Computer Science \\
{\smaller[1] Spring 2020} \\

\vspace {0.25in}

Sections 7.1 and 7.2
\end{center}

\vspace{0.25in}

{\smaller[1]
\textbf{Name:} \hrulefill
\hspace{1em}
\textbf{Username (abc123):} \rule{1.25in}{0.4pt}

\null \textbf{Recitation (circle one):}
\hfill
}

\begin{enumerate} % start main problems

% PROBLEM 1 (Respectively: #11, #9, and #15 from section 7.1)

\item For the following problems, assume you have a standard deck of 52 cards (4
suits for each of 13 ranks). Find the probability that a five-card poker hand:
%
\begin{enumerate}[itemsep=\fill,after=\vfill\vfill] % start subproblems

\item contains the two of diamonds, the three of spades, the six of hearts, the
ten of clubs, and the king of hearts.

\begin{answer}
There is one way to choose this hand, and there are $C(52, 5)$ ways to choose a
hand of 5. So the answer is $\frac{1}{C(52, 5)}$
\end{answer}

\item does not contain the queen of hearts.

\begin{answer}
There are $C(51, 5)$ ways to build a hand without the queen of hearts, and
$C(52, 5)$ total ways to build a deck. So, our answer is $\frac{C(51, 5)}{C(52,
5)}$
\end{answer}

\item contains two pairs. That is, the hand has two cards of one rank, two cards of a second rank, and a fifth card of a third rank.

\begin{answer}
There are $C(13,2)$ ways to choose the ranks for the two pairs. There are
$C(4,2)$ ways to select the suits for one pair, and $C(4,2)$ ways to choose the
other. There are $C(44, 1)$ ways to pick the fifth card, as there will be 44
cards left that are of neither kind already picked. Thus, the answer is
$\displaystyle\frac{C(13,2) \cdot C(4,2) \cdot C(4,2) \cdot C(44, 1)}{C(52, 5)}$
\end{answer}

\end{enumerate} % end subproblems


% PROBLEM 2 (7.1 #21)

\item What is the probability that a fair die never comes up an even number when
it is rolled six times?

\begin{answer}
Another way of wording this problem is ``What are the odds that you roll six odd
numbers in a row?'' Assuming this is a six-sided die, this means there are three
possible odd numbers for every roll. The number of ways we can roll six odd
numbers in a row is $3^6$. Then, the number of total ways to roll the die is
$6^6$. Thus, our answer is $\frac{3^6}{6^6}$, which simplifies to
$\frac{1}{64}$.
\end{answer}

\vfill

\newpage


% PROBLEM 3 (a is 7.1 #25, b is original/adaptation)

\item Consider a lottery where six winning numbers are picked from the range of
1 to 50, inclusive. The order of the numbers does not matter. Find the
probability of:
%
\begin{enumerate}[itemsep=\fill,after=\vfill] % start subproblems

\item picking all six winning numbers

\begin{answer}
There is only one possible winning combination, and $C(50, 6)$ ways to choose a
combination, so the answer is $\frac{1}{C(50, 6)}$
\end{answer}

\item picking exactly three of the winning numbers

\begin{answer}
There are $C(6,3)$ ways of picking three of the six winning numbers and
$C(44,3)$ ways of picking three non-winning numbers from the remaining pool of
possibilities. Since the total possible numbers are the same, the answer is
$\frac{C(6,3) \cdot C(44,3)}{C(50, 6)}$
\end{answer}

\end{enumerate} % end subproblems


% PROBLEM 4 (7.2 #29)

\item A group of 6 people go out to lunch, and decide to play the game ``odd
person out'' to determine who will pay. Each person flips a fair coin. If there
is a person whose outcome is not the same as that of any other member of the
group, this person has to pay for everyone's lunch. For example, if one person
gets tails and everyone else gets heads, the person with tails pays. What is the
probability that there is an odd person out after the coins are flipped once?

\begin{answer}
We start by counting the number of permutations where a loss occurs. There are
two ways to lose this game: you get heads and everyone else gets tails, or you
get tails and everyone else gets heads. One losing permutation is only the first
person gets tails, in another only the second person gets tails, etc.\ until we
get to the sixth person. That is six losing permutations. Plus, six more for the
cases where one person flips heads and the other five flip tails. That's 12
losing permutations. The total number of possible permutations is $2^6$. Our
answer is $\frac{12}{2^6}$, which simplifies to $\frac{3}{16}$
\end{answer}

\vfill
\vfill

\newpage


% PROBLEM 5 (a is 7.2, #25 and b is original/adaptation)

\item What is the conditional probability that a randomly-generated bit string
of length four contains at least two consecutive 0s, given that:
%
\begin{enumerate}[itemsep=\fill,after=\vfill] % start subproblems

\item the first bit is a 1?

\begin{answer}
To find conditional probability, we use the formula $\frac{P(E \intersect
F)}{P(F)}$. Let $E =$ ``the bit string has at least 2 consecutive 0's'', and let
$F =$ ``the first bit of the bit string is a 1.'' By writing out all cases, we
find that there are 3 elements in the set $E \intersect F$, which is \{1000,
1100, 1001\}. Also, we can say there are 16 total bit strings possible ($2^4$).
Thus, $P(E \intersect F) = \frac{3}{16}$ There are $2^3$ cases in the set of
$F$, because only the first bit of the string is decided, while the last three
can be 0 or 1. Thus $P(F)$ = $\frac{8}{16}$ Our final answer, simplified, is
$\frac{3}{8}$.
\end{answer}

\item The third bit is a 0?

\begin{answer}
Similar strategy to the previous problem, the only difference is in counting the
number of items in the set $E \intersect F$, which is \{1100, 1000, 0000, 1001,
0001, 0100\}, i.e., six items. Final, simplified answer is $\frac{3}{4}$
\end{answer}

\end{enumerate} % end subproblems


% PROBLEM 6 (Original problem)

\item Find the probability that a family with four children has two or three
girls, if the chance of having a boy is 30\%. Assume the sexes of each child are
independent.

\begin{answer}
This problem can be solved with the Bernoulli formula and the sum rule. $C(4, 2)
\cdot (0.7)^2 \cdot (0.3)^2 + C(4,3) \cdot (0.7)^3 \cdot (0.3)^1 = 0.2646 +
0.4116 = 0.6762$
\end{answer}

\vfill

\end{enumerate} % end main problems

\end{document}

