\documentclass[12pt]{article}
\usepackage[margin=1in]{geometry}
\usepackage{relsize}
\usepackage{amssymb}
\usepackage{amsmath}
\usepackage{enumitem}

\hyphenpenalty=10000
\setlength{\parindent}{0pt}

\newcommand\bicond{\mathrel{\leftrightarrow}}
\let\union\cup
\let\intersect\cap

\begin{document}

\begin{center}
CS 441: Discrete Structures for Computer Science \\
{\smaller[1] Spring 2020} \\

\vspace {0.25in}

Sections 7.1 and 7.2
\end{center}

\vspace{0.25in}

{\smaller[1]
\textbf{Name:} \hrulefill
\hspace{1em}
\textbf{Username (abc123):} \rule{1.25in}{0.4pt}

\null \textbf{Recitation (circle one):}
\hfill
}

\begin{enumerate} % start main problems

% PROBLEM 1 (Respectively: #11, #9, and #15 from section 7.1)

\item For the following problems, assume you have a standard deck of 52 cards (4
suits for each of 13 ranks). Find the probability that a five-card poker hand:
%
\begin{enumerate}[itemsep=\fill,after=\vfill\vfill] % start subproblems

\item contains the two of diamonds, the three of spades, the six of hearts, the
ten of clubs, and the king of hearts.

\item does not contain the queen of hearts.

\item contains two pairs. That is, the hand has two cards of one rank, two cards of a second rank, and a fifth card of a third rank.

\end{enumerate} % end subproblems


% PROBLEM 2 (7.1 #21)

\item What is the probability that a fair die never comes up an even number when
it is rolled six times?

\vfill

\newpage


% PROBLEM 3 (a is 7.1 #25, b is original/adaptation)

\item Consider a lottery where six winning numbers are picked from the range of
1 to 50, inclusive. The order of the numbers does not matter. Find the
probability of:
%
\begin{enumerate}[itemsep=\fill,after=\vfill] % start subproblems

\item picking all six winning numbers

\item picking exactly three of the winning numbers

\end{enumerate} % end subproblems


% PROBLEM 4 (7.2 #29)

\item A group of 6 people go out to lunch, and decide to play the game ``odd
person out'' to determine who will pay. Each person flips a fair coin. If there
is a person whose outcome is not the same as that of any other member of the
group, this person has to pay for everyone's lunch. For example, if one person
gets tails and everyone else gets heads, the person with tails pays. What is the
probability that there is an odd person out after the coins are flipped once?

\vfill
\vfill

\newpage


% PROBLEM 5 (a is 7.2, #25 and b is original/adaptation)

\item What is the conditional probability that a randomly-generated bit string
of length four contains at least two consecutive 0s, given that:
%
\begin{enumerate}[itemsep=\fill,after=\vfill] % start subproblems

\item the first bit is a 1?

\item The third bit is a 0?

\end{enumerate} % end subproblems


% PROBLEM 6 (Original problem)

\item Find the probability that a family with four children has two or three
girls, if the chance of having a boy is 30\%. Assume the sexes of each child are
independent.

\vfill

\end{enumerate} % end main problems

\end{document}

