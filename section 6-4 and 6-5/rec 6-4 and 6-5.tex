\documentclass[12pt]{article}
\usepackage[margin=1in]{geometry}
\usepackage{relsize}
\usepackage{amssymb}
\usepackage{amsmath}
\usepackage{enumitem}

\hyphenpenalty=10000
\setlength{\parindent}{0pt}

\newcommand\bicond{\mathrel{\leftrightarrow}}
\let\union\cup
\let\intersect\cap

\begin{document}

\begin{center}
CS 441: Discrete Structures for Computer Science \\
{\smaller[1] Spring 2020} \\

\vspace {0.25in}

Recitation on 6.3, 6.4, 6.5
\end{center}

\vspace{0.25in}

{\smaller[1]
\textbf{Name:} \hrulefill
\hspace{1em}
\textbf{Username (abc123):} \rule{1.25in}{0.4pt}

\null \textbf{Recitation:}
\hspace{2em}
Thursday 12:00--12:50 \hfill
}

\begin{enumerate} % start main problems

% PROBLEMS 5 and 6 from 7th Edition - section 6.3

\item Find the value of each of these quantities.

\begin{enumerate}

    \item $P(6, 3)$
    \item $C(5, 1)$
    \item $P(8, 5) + C(5, 3)$

\end{enumerate}

\vspace{1.5in}

% PROBLEM 19 from 7th Edition - section 6.3

\item A coin is flipped 10 times where each flip comes up either heads or tails. How many possible outcomes

\begin{enumerate}

    \item are there in total?
    \item contain exactly two heads?
    \item contain at most three tails?

\end{enumerate}

\newpage


% PROBLEM 3 from 7th Edition - section 6.4

\item Find the expansion of of the following binomials.

\begin{enumerate}

    \item $(x+y)^{3}$
    \item $(s+t)^{5}$

\end{enumerate}

\vspace{2in}
% PROBLEM 4 inspired by lecture slides for Pascal's Identity

\item Compute the following values using Pascal's Identity for $C(n, r)$.

\begin{enumerate}

    \item $C(6, 4)$
    \item $C(7, 5)$

\end{enumerate}

\newpage


% PROBLEM 3 from 7th Edition - section 6.5 (slightly tweaked)

\item How many strings of six letters are there

\begin{enumerate}

    \item if letters are replaced?
    \item if letters are NOT replaced?

\end{enumerate}

\vspace{2in}

%Problems 23, 21,  from 7th Edition - section 6.5 and part (c) inspired by lecture
\item How many ways are there to

\begin{enumerate}

    \item distribute six indistinguishable objects into nine distinguishable boxes?
    \item distribute six distinguishable balls into four indistinguishable boxes?

\end{enumerate}

\end{enumerate} % end main problems

\newpage

\begin{center}
    FORMULAS
\end{center}

\begin{align}
    &(x+y)^{n} = \sum_{j=0}^{n} C(n, j) x^{n-j} y^{j} \\
    &C(n, 0) = C(n, n) = 1 \text{ and } C(n+1, k) = C(n+1, k-1) + C(n, k) \\
    &\sum_{j=1}^{k} S(n, j) = \sum_{j=1}^{k} \frac{1}{j!} \sum_{i=0}^{j-1} (-1)^{i} C(j, i) (j-i)^{n} \\
    &P(n, r) = \frac{n!}{(n-r)!} \\
    &C(n, r) = \frac{n!}{(n-r)! r!} \\
\end{align}

\end{document}

